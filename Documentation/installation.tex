\chapter{Installation}
\section{Requirements}

For installing \emph{User Data Defence} on your machine you need to
download from system repository some packages:

\begin{itemize}
\item
  SELinux;
\item
  Audit daemon must be installed and activated;
\item
  QT4 development package including D-Bus support;
\item
  audit-libs-devel;
\item
  libselinux-devel;
\item
  dbus-devel.
\end{itemize}

\section{Warning}

Before installation \emph{User Data Defence} switch your \emph{SELinux}
in \emph{Permissive} mode. Type in root console:

\texttt{setenforce 0}

Then edit \texttt{/etc/selinux/config}: replace
\texttt{SELINUX=disabled} or \texttt{SELINUX=enforcing} with
\texttt{SELINUX=permissive}.

\section{Installation}

For installation type in your console:

\texttt{git clone git@github.com:Hramchenko/userdatadefence.git}

\texttt{cd ./userdatadefence}

\texttt{qmake (or qmake-qt4 in Fedora)}

\texttt{make}

\texttt{make install}

For applying changes reboot your computer. If \emph{UDDTray} not started
automatically you could start it from console or system menu. At first
start you must receive many alerts, otherwise check that you
\emph{SELinux} is in Permissive mode, then check that \emph{UDDaemon}
and \emph{UDDBus} are started correctly. If \emph{UDDaemon} was not
started you can start it manually from root console:

\texttt{UDDaemon \&}

If you can't find \emph{UDDBus} in process list check contents of file
\texttt{/etc/audisp/plugins.d/UDDBus.conf} and restart \emph{auditd}.

If \emph{UDDaemon} and \emph{UDDBus} were started check their contexts:

\texttt{ps axZ \textbar{} grep UDD}

\texttt{system\_u:system\_r:user\_data\_defence\_bus\_t:s0 948 ? S\textless{}   0:00 /usr/sbin/UDDBus}

\texttt{system\_u:system\_r:user\_data\_defence\_daemon\_t:s0 1468 ? S   0:00 /usr/sbin/UDDaemon}

If they have another types check that \emph{udd} policy was loaded:

\texttt{semodule -l \textbar{} grep udd}

\texttt{udd     1.0.0}

If policy was not loaded install it with \texttt{semodule -i udd.pp}
command. You may need to customize it for your system.

\section{Post installation}

When your system worked correctly (you don't receive false-positive AVC
alerts) you could try to change context of your user. Switch SELinux in
\emph{Permissive} mode and type:

\texttt{semanage login -a -s staff\_u your\_login\_name}

Then append some aliases in \textasciitilde{}/.bash\_rc profile:

\texttt{alias su="sudo -u sysadm\_u -r sysadm\_r -t sysadm\_t -u root bash"}

Start new session. You will receive new SELinux alerts. Process it with
\emph{UDDTray} generation policy tool. After some days of system usage,
when you don't get false-positive alerts, you could try to switch
SELinux in \emph{Enforcing} mode (mode with intrusion prevention).

Open graphical console (\emph{konsole}, \emph{gnome-terminal}\ldots{})
execute \texttt{su} or \texttt{sudo bash} command. Temporary switch
\emph{SELinux} in \emph{Enforcing} mode. You will get new alerts. Append
it in a policy.

\textbf{Warning!} Some AVC messages are not shown in alerts list because
there are not processed by \emph{auditd} service (such as D-Bus
messages). They are shown only in \texttt{/var/log/messages} file. You
could generate policy for it with \emph{Generate policy for
/var/log/messages} option in \emph{UDDTray}.

\emph{Enforcing} mode will resetting after computer restart. So if your
system worked properly in that mode set \texttt{SELINUX=enforcing} in
\texttt{/etc/selinux/config} file.

\section{Troubleshooting}

If your system doesn't work correctly and you don't have any audit
messages or a system messages in \texttt{/var/log/messages} file try to
rebuild SELinux policy with disabled \emph{dontaudit} rules. In root
console type \texttt{semodule -D}

When problem was resolved you could enable rules with
\texttt{semodule -B} command.
%
%\href{https://github.com/Hramchenko/userdatadefence/wiki/Creating-new-policy}{Go
%to the next page}