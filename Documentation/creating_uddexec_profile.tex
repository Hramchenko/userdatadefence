\section{Creating UDDExec profile}

\emph{UDDExec} is an application launcher utility. It executes the
calling program in depending on the mode selected in \emph{UDDTray}. Now
we configure \emph{UDDTray} to select modes of \emph{Libre Office} policy (page \pageref{creating_new_policy}).

Open preferences of \emph{UDDTray}:
\emph{Edit-\textgreater{}Preferences}, and select \emph{Applications}
group.

\illustrationWithCaption[0.55]{exec_groups}{Applications groups}

Append new group \emph{Office} by activating plus button. Enter group
name, and text which will be shown in \emph{UDDTray} menu. Select icon
for this group. Go to \emph{Group modes} tab.

\illustrationWithCaption[0.55]{exec_modes}{Applications groups modes}

Append modes \emph{Unclassified} and \emph{Secret} in group
\emph{Office}, select its names, menu texts, icons. Open
\emph{Applications} tab.

\illustrationWithCaption[0.55]{exec_applications}{Enforced applications}

Last step is to append application with name \emph{libreoffice} and
command \texttt{libreoffice} in mode \emph{Unclassified}, and append
application with name \emph{libreoffice} and command
\texttt{libreoffice\_secret} in mode \emph{Secret}.

Now, if you select mode \emph{Unclassified} in \emph{UDDTray} menu,
application \emph{libreoffice} will called when you type in users
console \texttt{UDDExec libreoffice -writer}, and
\emph{libreoffice\_secret} called on command
\texttt{UDDExec libreoffice -writer} in \emph{Secret} mode.

\illustrationWithCaption[0.55]{exec_tray}{Selection of UDDExec mode}

So if \texttt{/usr/bin/libreoffice} has type \texttt{libre\_office\_t},
and \texttt{/usr/bin/libreoffice\_secret} has type
\texttt{libre\_office\_secret\_t} you could call program with single
command but different domains.

You could replace \texttt{.desktop} file of \emph{Libre Office} to your
own file with command \texttt{UDDExec libreoffice}.

