\chapter{Introduction}

\emph{User Data Defence} is a system based on \emph{SELinux}, which provides protection for your documents when user space applications (such as web browser, PDF viewer) were attacked.

\illustrationWithCaption[0.6]{mainwindow}{Main Window}

\section{How does it works?}
\emph{User Data Defence} includes set of template policies, which makes
process of creation \emph{SELinux} specifications for user mode
applications simple as never before. Now you could protect documents on
your workstation against user mode viruses or program errors.

\section{Features}
This program provides advanced \emph{SELinux} events
notification on the Desktop. You could choose notification images and
text according to event type or reg exp pattern.

\illustrationWithCaption[0.6]{libre_notification}{Alerts notification}

\emph{User Data Defence} provides you an opportunity to specify a
security policy for individual applications, depending on the type of
information which need to be processed.

\illustrationWithCaption[0.6]{exec_tray}{UDDTray: Selection mode}

One of the main goals for creating User Data Defence was to create
replacement of \emph{setroubleshootd} with low CPU usage. Now when
system received many alerts in a short period of time CPU usage is not
so high.

