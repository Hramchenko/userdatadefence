\chapter{Architecture}

\emph{User Data Defence} consist of five components: \emph{UDDBus},
\emph{UDDaemon}, \emph{UDDTray}, \emph{UDDExec} and \emph{UDDPolicy}.

\section{UDDBus}

\emph{UDDBus} is an \emph{auditd} interaction utility. It reads from the
input stream of audit events daemon, filters AVC messages from stream
and transmits them via D-Bus Service to UDDaemon (this utility is based
on code of \emph{sedispatch}, written by D. Walsh).

\section{UDDaemon}

\emph{UDDaemon} is a daemon of SELinux messages. Daemon receives data
from \emph{UDDBus}, accumulates it and provides a data storage. It sends
information about new security events to \emph{UDDTray} in a real time.

\section{UDDTray}

\emph{UDDTray} is a userland component of \emph{User Data Defence}. It
is a graphical application which running in the system tray.
\emph{UDDTray} performs SELinux alerts through the system notification
service \emph{KNotify}. It provides an interface of controlling modes of
access control system.

\section{UDDExec}

\emph{UDDExec} is an application launcher utility. It selects the
calling program in depending on the mode selected in \emph{UDDTray}.
This utility provide you an opportunity to specify a security policy for
individual applications, depending on the type of information which need
to be processed.

\section{UDDPolicy}

\emph{UDDPolicy} is a set of SELinux policy templates for applications
with graphical user interface. It provides new macro functions which
helps in rapid policy development.
