\chapter{Writing SELinux policy}
\section{Creating new policy}\label{creating_new_policy}

This example demonstrates how to create policy for application with
graphical interface \emph{Libre Office}.

\subsection{Creating policy files}

For creating new policy select \emph{File-\textgreater{}New Policy} in
\emph{UDDTray} menu. In \emph{New Policy} window select \emph{Gui
application policy}. Then enter existing directory for policy files and
its name - \texttt{libre\_office}.

\illustrationWithCaption[0.57]{new_policy}{Policy creation window}

When necessary fields was filled \emph{Create} button will activated.
Press it. \emph{UDDTray} generates \texttt{libre\_office.te},
\texttt{libre\_office.if}, \texttt{libre\_office.fc}. Also it creates
\texttt{udd.te}, \texttt{udd.fc}, \texttt{udd.if} and \texttt{Makefile},
if there are not exists.

\subsection{Customizing .te file}

First line describes policy module name and its version:

\texttt{policy\_module(libre\_office,1.0.0)}

Next four lines contains external types and roles which are used inside
a policy:

\texttt{require \{}

\texttt{type staff\_t;}

\texttt{role staff\_r;}

\texttt{\}}

Macro function \texttt{udd\_gui\_application\_create(libre\_office);}
creates:

\begin{itemize}
\item
  \texttt{libre\_office\_t} - domain type;
\item
  \texttt{libre\_office\_exec\_t} domain entry point - type of
  executables which can be used to enter a domain;
\item
  \texttt{libre\_office\_config\_dir\_t},
  \texttt{libre\_office\_config\_file\_t} - types of files and
  directories with users configuration files (usually located at home
  folder).\\
\item
  \texttt{libre\_office\_tmpfs\_t} - type of temporary files.
\end{itemize}

Line
\texttt{udd\_gui\_application\_access(libre\_office, staff\_r, staff\_t);}
allow user with role \texttt{staff\_r} and type \texttt{staff\_t} access
to domain \texttt{libre\_office\_t}. It provides domain auto transition
from \texttt{staff\_t} to \texttt{libre\_office\_t}, when user executes
file with context \texttt{libre\_office\_exec\_t}. Also this instruction
allows D-Bus communication between user domain and
\texttt{libre\_office\_t}.

Command
\texttt{udd\_gui\_application\_append\_special\_domain(libre\_office, secret);}
creates domain \texttt{libre\_office\_secret\_t} with entry point
\texttt{libre\_office\_secret\_exec\_t} and creates types for secret
files:

\begin{itemize}
\item
  \texttt{libre\_office\_secret\_file\_t} secret files type;
\item
  \texttt{libre\_office\_secret\_dir\_t} directories type. Any file
  written in that directory by application with domain
  \texttt{libre\_office\_t} will have context
  \texttt{libre\_office\_secret\_file\_t}, any creating folder will have
  type \texttt{libre\_office\_secret\_dir\_t};
\end{itemize}

Line
\texttt{udd\_gui\_application\_special\_domain\_access(libre\_office, secret,  staff\_r, staff\_t);}
allows user with role \texttt{staff\_r} and domain \texttt{staff\_t}
access to \texttt{libre\_office\_secret\_t}. It creates domain auto
transition, and allows their D-Bus communication;

\subsection{Editing .fc specification}

Lets edit \texttt{libre\_office.fc}.

Change line
\texttt{/path/to/application -{}- system\_u:object\_r:libre\_office\_exec\_t:s0}
to
\texttt{/usr/bin/libreoffice -{}- system\_u:object\_r:libre\_office\_exec\_t:s0}.

Copy \texttt{libreoffice} to \texttt{libreoffice\_secret}:
\texttt{cp /usr/bin/libreoffice /usr/bin/libreoffice\_secret}, and
replace line \texttt{/path/to/application\_secret} with
\texttt{/usr/bin/libreoffice\_secret} at the second line.

Next, replace
\texttt{HOME\_DIR/\textbackslash{}.libre\_office\_config(/.*)} with
\texttt{HOME\_DIR/\textbackslash{}.libreoffice/}.

Let we have directory with secret files at
\texttt{/home/secret\_documents}, so we need to replace
\texttt{HOME\_DIR/\textbackslash{}.libre\_office\_secret\_dir(/.*)?}
with \texttt{/home/secret\_documents(/.*)?} at the last two lines of
configuration file.

After all modifications you receive as a result:

\texttt{/usr/bin/libreoffice -{}- system\_u:object\_r:libre\_office\_exec\_t:s0}

\texttt{/usr/bin/libreoffice\_secret -{}- system\_u:object\_r:libre\_office\_secret\_exec\_t:s0}

\texttt{HOME\_DIR/\textbackslash{}.libreoffice(/.*)? -d system\_u:object\_r:libre\_office\_config\_dir\_t:s0 HOME\_DIR/\textbackslash{}.libreoffice(/.*)? -{}- system\_u:object\_r:libre\_office\_config\_file\_t:s0}

\texttt{/home/secret\_documents(/.*)? -d staff\_u:object\_r:libre\_office\_secret\_dir\_t:s0}

\texttt{/home/secret\_documents(/.*)? -{}- staff\_u:object\_r:libre\_office\_secret\_file\_t:s0}

\subsection{Policy installation}

For installation type in your root console:

\texttt{cd /path/to/policy/folder}

\texttt{make}

\texttt{semodule -i libre\_office.pp}

Next step is to change files contexts:

\texttt{restorecon -R /usr/bin/libreoffice* /home/*/.libre\_office /home/secret\_documents}

